
%----------------------------------------------------------------------------------------
% WORKING WITH SOURCES
%----------------------------------------------------------------------------------------
\hypertarget{sources}{}


\chapter{Working with Sources}

\begin{quote} \small [A] quotation is a handy thing to have about, saving one the trouble of thinking for oneself, always a laborious business.

\textemdash A.A. Milne
\end{quote}

\begin{quote} \small The ugly fact is books are made out of books.

\textemdash Cormac McCarthy

\end{quote}

Academic writing always involves integrating the thinking of others into your own writing. There are only three ways that the words and ideas
of others may appear in your writing: \hyperlink{summary}{\color{Ahrenge}{summary}}, \hyperlink{paraphrase}{\color{Ahrenge}{paraphrase}}, and \hyperlink{quotation}{\color{Ahrenge}{quotation}}. Writing an academic paper requires a mastery of
all three skills. It is critical to always give credit to the other
authors whose ideas or words you borrow. Failing to do so may result in the
accusation of \hyperlink{plagiarism}{\color{Ahrenge}plagiarism}. The way scholars 
avoid plagiarism is by using \textbf{signal phrases} and \hyperlink{citation}{\color{Ahrenge}{citations}}.

\hypertarget{signalphrase}{}
\section{Using signal phrases}


\subsection{What are they?}

Signal phrases are words that tell your readers that you are borrowing words or ideas from a source. The borrowed material could be quoted, summarized, or paraphrased. The signal phrase, as the name suggests, tells your reader that you are about to begin borrowing; after you have presented the borrowed material, conclude with a \hyperlink{citation}{\color{Ahrenge}{citation}} to tell your reader that the borrowing has concluded. Using signal phrases and citations to bookend your borrowings from other texts helps you avoid \hyperlink{plagiarism}{\color{Ahrenge}{plagiarism}}, organize your writing, and help your readers understand how your views relate to the views of the other writers you present.

\subsection{Why use them?}
 \begin{itemize}

\item They make it clear that you are transitioning from your own ideas and
writing to the ideas and writing of another. This makes your paper more
coherent.

\item They make it clear when you have begun to \hyperlink{paraphrase}{\color{Ahrenge}{paraphrase}} or \hyperlink{summary}{\color{Ahrenge}{summarize}}.
Unlike quotations, paraphrases and summaries are not formatted with quotation
marks; therefore, it is difficult for readers to know when or where you have
begun to paraphrase or summarize unless you include these phrases.

\item They make the tone of your paper more academic and authoritative.

\item They compel you to articulate how your ideas relate to those you
have borrowed from others. This will direct your attention to the precise
ways in which authors agree or disagree with you or each other, and allow
you to make these intersections clear to your reader.

\item Using these signal phrases will help you to avoid \hyperlink{plagiarism}{\color{Ahrenge}plagiarism}.
 \end{itemize}

\subsection{How do you construct a signal phrase?}

\begin{itemize}
\item \textbf{Use the author's name}. The first time you mention an author,
include the
author's full name, the title of his or her work, and perhaps a brief
statement indicating the
author's credentials. Once you have introduced an author in your paper,
only use his or
her last name if you mention him or her again.

\item \textbf{Use a strong verb to characterize what the author has done.} See the list
below for suggestions. Be sure to pick the verb which most precisely
articulates the
author's action:

\begin{quote}
asserts, argues, believes, claims, emphasizes, insists, observes, reports,
suggests, acknowledges, admires, agrees, corroborates, endorses, extols,
praises, verifies, illustrates, expands on, rejects, complicates, contends, contradicts,
denies, disagrees, refutes, questions, warns, proposes, implores, exhorts, demands,
calls for, recommends, urges, advocates, wonders, asks, rejects, encourages.
\end{quote}

\end{itemize}

\subsection{Example of a signal phrase use}

Here is an example of a student properly using signal phrases and citations to indicate borrowings from other sources:

\begin{quote}

A number of views exist that attempt to explain the nature and origins of Taoist religious practices. \hl{In his book \emph{The Tao Practice}, scholar Dean Anderson argues} that "Taoism emerged during a period of unprecedented struggle, deprivation, and suffering in the fourth century BCE" (89). \hl{This historical condition, he explains,} resulted in religious and cultural practices and beliefs that valued asceticism and practiced detachment from desire (12). However, other historians disagree. \hl{In his \emph{Up and Down of the Tao}, Li Chang argues} that Taoism as we know it was largely a creation of the 17th-century, a time of relative prosperity, radical socio-political change, and modernization in China (5). \hl{For Chang,} Taoist asceticism was actually a \emph{rejection} of this tumultuous cultural transformation\textemdash an expression of nostalgia for a simpler time in the ancient Chinese past (22). But which is the correct answer? Did Taoism emerge in a time of poverty or a time of plenty? In my view, Taoism . . .

\end{quote}


\noindent Notice here now the author of this paragraph is careful to distinguish his voice from the two source texts he is using. The student announces that he is borrowing words or ideas (in the form of quotations, summary, or paraphrase) with a signal phrase and the author's name, then ends the borrowing with a citation. The paragraph concludes with the student transitioning from the source texts to his own thoughts, posing a series of questions that he will try to answer. 


\hypertarget{quotation}{}
\section{Quotation}

\subsection{When should I quote something?}

Quotation is the inclusion of another author's exact wording in your own writing. While quotation is a critical element of all academic writing, you must be judicious in its use. Only quote when the rhetorical situation 
requires it. The overuse of quotation can make you appear lazy or lacking in confidence. That said, there are moments when quotations are entirely justified. For example:

\begin{itemize}

\item When you are interpreting literature such as a poem or novel, the specific language used in the text is the subject of your essay. That is to say, your argument is about the meaning of the exact words chosen by the author for his or her literary work. It is critical in these instances to use quotations from the literary text and then explain what those words mean to your audience\textemdash a process known as "\hyperlink{closereadingessay}{\color{Ahrenge}{close reading}}" in literary studies. 


\item When you are making an argument it is often helpful to use the words of known authorities to help make your case. While you may not be a doctor, a physicist, or professional journalist, you may use their words and arguments to help give credibility to your ideas. While using the exact words of these important authorities can be rhetorically effective, make sure to use them prudently. If your essay becomes a mere tissue of quotation, your authority as an author is undermined. The strongest voice in your essay should be your own. Allow these other voices to be briefly heard; don't allow them to drown out your own voice.

\item When that the source text contains language that is memorable, beautiful, or particularly apt, quotation is 
justified. If you feel that summarizing or paraphrasing would do violence to the original language, using a quotation is often the best choice.

\item A quotation is often necessary when you describe legal discourse (such as a law or court ruling) where words cannot be paraphrased or summarized without altering the meaning and effect of the legal language.

\end{itemize}

\noindent For most other circumstances, \hyperlink{summary}{\color{Ahrenge}{summary}} or \hyperlink{paraphrase}{\color{Ahrenge}{paraphrase}} of the original language is best.

\subsection{How do I integrate quoted material?}

 \begin{itemize}
\item Use a \hyperlink{signalphrase}{\color{Ahrenge}{signal phrase}} to introduce the quoted passage.

\item Use quotation marks.

\item Provide a \hyperlink{citation}{\color{Ahrenge}{citation}} in your chosen format, such as MLA or Chicago.

\item If necessary, use \hyperlink{ellipsis}{\color{Ahrenge}{ellipsis}} or \hyperlink{brackets}{\color{Ahrenge}{brackets}} to alter the source,
satisfy grammar, or provide clarifying information.

\end{itemize}


\subsection{What should I avoid?}

\begin{itemize}
\item \textbf{Avoid excessive use of quotation}. If you quote too often it can
make it appear that you have not fully read or understood the source material.
It may also make your writing appear lazy and thoughtless.

\item \textbf{Avoid excessive use of block quotation}. Block quotations should
be rare; reserve them for special language that you believe cannot be
summarized or paraphrased. 

\item \textbf{Avoid inserting a quote within your writing without providing
your commentary or explanation}. Explain to your audience what your quotes mean and connect them to your broader argument so that the reader will understand how to interpret them.

\item \textbf{Avoid inserting quotations without signal phrases}. Quotations
should be introduced and woven into your own writing. They should rarely stand alone.

\end{itemize}

\subsection{What if the original quotation has an error?}

Occasionally you will want to quote a text that contains an error of some sort. Perhaps the author used the wrong word or there is a misspelling or grammatical error. In these cases, you may want to indicate to your readers that the error exists in the original text and is not a sloppy accident of your own making. To communicate this to your readers, use the Latin term \emph{sic}, or "thus," next to the offending word or error. For example:\medskip

\begin{itemize}
\item According to the report, "The children were told to make there [sic] beds" (98). 
\end{itemize}



\subsection{Example of a quotation}

\begin{tcolorbox}[enhanced,width=4.2in,left=.3in, right=.3in,
   drop fuzzy shadow southeast,
    boxrule=0.4pt,sharp corners,colframe=black!80!black,colback=white!10]

\medskip

{\small
\begin{doublespacing}
\textbf{\hl{Source Text}}
\smallskip

\hspace{.5cm}At the heart of the environmentalist worldview is the conviction that human physical and spiritual health depends on sustaining the planet in a relatively unaltered state. Earth is our home in the full, genetic sense, where humanity and its ancestors existed for all the millions of years of their evolution. Natural ecosystems\textemdash forests, coral reefs, marine blue waters\textemdash maintain the world exactly as we would wish it to be maintained. Our body and our mind evolved precisely to live in this particular planetary environment and no other. When we debase the global environment and extinguish the variety of life, we are dismantling a support system that is too complex to understand, let alone replace, in the foreseeable future (238).

\bigskip

\noindent\textemdash E.O. Wilson, “\href{https://doi-org.dartmouth.idm.oclc.org/10.1016/0303-2647(93)90052-E}{Is Humanity Suicidal?}”

\bigskip

\end{doublespacing}}

\end{tcolorbox}

\subsubsection*{Sample quotations from the source text:}

\begin{itemize}

\item In a recent essay, scientist E.O. Wilson considers a dark truth about humanity: "we are dismantling a support system that is too complex to understand, let alone replace, in the foreseeable future" (238).

\item "At the heart of the environmentalist worldview," claims scientist E.O. Wilson, "is the conviction that human physical and spiritual health depends on sustaining the planet in a relatively unaltered state" (238).

\item One important biologist insists that if we "debase" the planet we risk "dismantling a support system" that is too complicated to understand or replace (Wilson 238).

\end{itemize}

\hypertarget{summary}{}
\section{Summarizing}
In a summary you present the ideas of another writer in a condensed form. The
length of a summary is dictated by your rhetorical needs, however \emph{they are
always shorter than the original text}. For example, the summary of a large
book could be 20 pages, one paragraph, or one sentence. Although a summary
sacrifices specificity and detail in the interest of brevity, it must always
remain a faithful representation of the original text.


\subsection {Why are summaries important?}

Summary is one of the central skills needed for academic writing. Summary often appears
in an academic essay's introduction to provide readers with background information or historical context. In academic writing, summary is often used to explain a complex scholarly conversation that the writer plans to enter with his or her essay. An excellent summary of this broader scholarly conversation goes far to establish you as a knowledgeable authority with your readers\textemdash someone whose views should be trusted and considered. Summary is particularly useful when we make use of secondary sources in our writing. If we want to use or introduce another source in our own writing, we use summary to inform our audience about the arguments and ideas contained within it. We also make significant use of summary in the complex work of \hyperlink{synthesis}{\color{Ahrenge}{synthesis}}, where we explain how two or more texts relate to one another. 

\subsection{How do I incorporate summaries?}

\begin{itemize}
\item Since summaries do not use quotation marks, you must take care to
indicate to your readers that you are borrowing from the work of others. This is primarily accomplished
through the use of a \hyperlink{signalphrase}{\color{Ahrenge}{signal phrase}} and a citation. As you move from your own writing to the summary of others, use a signal phrase to indicate this transition.

\item Think of the signal phrase and the citation as a way to bookend a borrowing from a source. The signal phrase alerts readers that you are about to borrow from another text; the \hyperlink{citation}{\color{Ahrenge}{citation}} is used to show that the borrowing has concluded. End the summary with an appropriate \hyperlink{citation}{\color{Ahrenge}{citation}}, noting the page(s) summarized.
\end{itemize}

\subsection{What should I avoid?}
\begin{itemize}

\item \textbf{Avoid plagiarizing}. Remember, summarized material
is still borrowed material, even though you have greatly condensed it and
put it entirely in your own words. Make sure that any summarized material is 
introduced with a \hyperlink{signalphrase}{\color{Ahrenge}{signal phrase}} and concluded with a \hyperlink{citation}{\color{Ahrenge}{citation}}.
\end{itemize}


\subsection{Example of a summary}


\begin{tcolorbox}[enhanced,width=4.2in,left=.3in, right=.3in,
   drop fuzzy shadow southeast,
    boxrule=0.4pt,sharp corners,colframe=black!80!black,colback=white!10]

\medskip

{\small
\begin{doublespacing}
\textbf{\hl{Source Text}}
\smallskip

\hspace{.5cm}When academic territories were parceled out in the early twentieth century,
anthropology got the tellers of tales and history got the keepers of written
records. As anthropology and history diverged, human differences that
hinged on literacy assumed an undeserved significance. Working with oral,
preindustrial, prestate societies, anthropologists acknowledged the power
of culture and of a received worldview; they knew that the folk conception
of the world was not narrowly tied to proof and evidence.  But with the
disciplinary boundary overdrawn, it was easy for historians to assume that
literacy, the modern state, and the commercial world had produced a different
sort of creature entirely\textemdash humans less inclined to put myth over reality,
more inclined to measure their beliefs by the standard of accuracy and
practicality. (35)

\bigskip

\noindent\textemdash Patricia Nelson Limerick, \href{http://libcat.dartmouth.edu/record=b1422593~S1}{\emph{The Legacy of Conquest}}.

\bigskip

\end{doublespacing}}

\end{tcolorbox}

\subsubsection*{Sample summary from the source text:}

\begin{quote}
As Patricia Nelson Limerick argues in \emph{The Legacy of Conquest},
historians have falsely assumed that literate societies with vibrant economies and
systems of governance were never beholden to myth or superstition (35).
\end{quote}

\hypertarget{paraphrase}{}
\section{Paraphrasing}

Think of paraphrase as a translation from English into English. It involves
taking
language from a source, putting it in your own words, and arranging it within
your own original
sentence structure(s). Unlike \hyperlink{summary}{\color{Ahrenge}{summary}}, which aims to reduce or distill an
idea, a paraphrase should
be similar in length to the original passage.

\subsection{Why are paraphrases important?}

Accurate paraphrase demonstrates mastery of your source materials and
indicates an author who is in control of his or her own writing and thinking. Whereas excessive
\hyperlink{quotation}{\color{Ahrenge}{quotation}} may reveal an uncertain or tentative author, paraphrase demonstrates control and
confidence. However, ensure that your paraphrases do justice to the original, or risk compromising
your authority with your readers.

\subsection{How do I incorporate paraphrases?}

\begin{itemize}
\item Like summaries, paraphrases do not use quotation marks. As a result,
you must take care to indicate to your readers that you are borrowing from
the work of others with a \hyperlink{signalphrase}{\color{Ahrenge}{signal phrase}} and \hyperlink{citation}{\color{Ahrenge}{citation}}. As you move
from your own writing to the paraphrase of others, use a signal phrase to
indicate this transition.

\item End the paraphrase with an appropriate citation.

\end{itemize}

\subsection {What should I avoid?}

\begin{itemize}

\item \textbf{Avoid plagiarizing}. Remember, paraphrased material
is still borrowed material, even though you have put it in your own words. Make sure that any paraphrased material is introduced with a \hyperlink{signalphrase}{\color{Ahrenge}{signal phrase}} and concluded with a \hyperlink{citation}{\color{Ahrenge}{citation}}.

\item \textbf{Avoid "patchwriting."} \hyperlink{patchwriting}{\color{Ahrenge}{Patchwriting}} is a process of merely changing a word or a phrase here or there. Instead, read the passage until you can put it aside and write your paraphrase without having to look back at it.
\end{itemize}

\subsection{Example of a paraphrase}

\begin{tcolorbox}[enhanced,width=4.2in,left=.3in, right=.3in,
   drop fuzzy shadow southeast,
    boxrule=0.4pt,sharp corners,colframe=black!80!black,colback=white!10]

\medskip

{\small
\begin{doublespacing}
\textbf{\hl{Source Text}}
\smallskip

\hspace{.5cm}When academic territories were parceled out in the early twentieth century,
anthropology got the tellers of tales and history got the keepers of written
records. As anthropology and history diverged, human differences that
hinged on literacy assumed an undeserved significance. Working with oral,
preindustrial, prestate societies, anthropologists acknowledged the power
of culture and of a received worldview; they knew that the folk conception
of the world was not narrowly tied to proof and evidence. But with the
disciplinary boundary overdrawn, it was easy for historians to assume that
literacy, the modern state, and the commercial world had produced a different
sort of creature entirely\textemdash humans less inclined to put myth over reality,
more inclined to measure their beliefs by the standard of accuracy and
practicality. (35)

\bigskip

\noindent\textemdash Patricia Nelson Limerick, \href{http://libcat.dartmouth.edu/record=b1422593~S1}{\emph{The Legacy of Conquest}.
}

\bigskip

\end{doublespacing}}

\end{tcolorbox}


\subsubsection*{Sample paraphrase from the source text:}

\begin{quote}
In \emph{The Legacy of Conquest}, Patricia Nelson Limerick argues that during the
early part of the last century the disciplines of anthropology and history
separated. While anthropology focused on unlettered and illiterate communities,
history became the study of societies who produced texts and records. Within
the field of anthropology, a firm belief developed that oral cultures were
characterized by mythological worldviews and superstitious beliefs; on
the other hand, historians improperly assumed that literate cultures were filled with
individuals who only used reason and evidence to guide their thinking (35).
\end{quote}




