%----------------------------------
% Common Sentence Errors
%-----------------------------------



\chapter{Common Sentence Errors}
\hypertarget{Top20}{}



A \href{http://www.jstor.org/stable/357695}{statistical study of student writing} performed in 1988 by scholars Andrea Lunsford and Robert Connors demonstrated that virtually all student
writing mistakes are limited to 20 formal errors. Eliminating these 
errors in your writing therefore offers the quickest path to error-free prose.

\setcounter{secnumdepth}{0}

\section{1. Missing comma after an introductory element}


\begin{itemize}
\item Similarly Freire argues that memorization as a form of education is dehumanizing. \ding{55}
\item Similarly, Freire argues that memorization as a form of education is dehumanizing. \ding{51}
\medskip
\item If the team wants to win they will have to practice more. \ding{55}
\item If the team wants to win, they will have to practice more. \ding{51}
\end{itemize}

\noindent Introductory words or clauses are usually set off with a comma.


\section{2. Wrong word} 
\begin{itemize}
\item The workmen assembled at the \hl{cite}. \ding{55}

\item The workmen assembled at the site. \ding{51}
\medskip
\item I have a bad case of \hl{ammonia}. \ding{55}

\item I have a bad case of pneumonia. \ding{51}
\end{itemize}


\section{3. Incomplete or missing documentation} 

\begin{itemize}
\item The response to the threat of terrorism should not be a curtailing of
freedoms. \ding{55}

\item The response to the threat of terrorism should not be a curtailing of
freedoms (Taylor 29). \ding{51}
\end{itemize}

\noindent Missing documentation for a quotation, summary, or paraphrase of another text
may result in charges of \hyperlink{plagiarism}{\color{Ahrenge}{plagiarism}}, a serious academic offense.

\section{4. Vague pronoun reference} 
\begin{itemize}
\item The teacher gave her notes to her. \ding{55}

\item The teacher gave her notes to Jane. \ding{51}
\end{itemize}


\section{5. Spelling error} 
\begin{itemize}
\item I \hl{definately} will be there. \ding{55}

\item I definitely will be there. \ding{51}
\end{itemize}


\section{6. Faulty Parallelism} 
\begin{itemize}
\item He was good at swearing, fighting, and liked to drink. \ding{55}
\item He was good at swearing, fighting, and drinking. \ding{51}
\smallskip

\item They were informed that they should not eat before swimming, that they should not eat sugar, and to do some exercises before bed. \ding{55}

\item They were informed that they should not eat before swimming, that they should not eat sugar, and that they should not skip exercise before bed. \ding{51}

\end{itemize}

\noindent Parallel structure involves using the same form of words or structural pattern when crafting a sentence. In the example above, the author uses gerunds for the first two verbs but then changes to the infinitive form for the last verb. Generally, parallel structure sounds much better to the ear than otherwise. 


\section{7. Unnecessary comma} 
\begin{itemize}
\item The legal language applies to carnivals, and to amusement parks.  \ding{55}

\item The legal language applies to carnivals and to amusement parks.  \ding{51}
\end{itemize}
\medskip
\begin{itemize}
\item The cemetery on the hill, is haunted. \ding{55}

\item The cemetery on the hill is haunted. \ding{51}
\end{itemize}
\medskip
\begin{itemize}
\item The man, who held the American flag, waved to us from the tour bus.  \ding{55}

\item The man who held the American flag waved to us from the tour bus.  \ding{51}
\end{itemize}

\noindent The phrase "who held the American flag" is a \emph{restrictive} element: a part of a sentence that is essential its meaning. This information identifies the particular man who waved from all the others on the bus. Restrictive elements are \emph{never} set off with commas. 


\section{8. Missing comma with a nonrestrictive element} 

\begin{itemize}
\item Jeff who owned the corporation was a big gambler and a cheat. \ding{55}

\item Jeff, who owned the corporation, was a big gambler and a cheat. \ding{51}
\end{itemize}

\noindent A \textbf{nonrestrictive element} is a part of a sentence that is not essential to its
meaning. Commas are used to set off these nonessential portions of the
sentence.


\section{9. Missing comma in compound sentence} 

\begin{itemize}
\item I've given him all that I own and I can't see myself giving more. \ding{55}

\item I've given him all that I own, and I can't see myself giving more. \ding{51}
\end{itemize}

\noindent A compound sentence contains two or more clauses that can stand alone as
complete sentences (otherwise known as "independent clauses"). However,
to connect them you must either use a semicolon or use a comma and
coordinating conjunction such as \emph{and}, \emph{but}, or \emph{yet}. Failing
to punctuate the compound sentence results in a \textbf{fused}, or \textbf{run-on}, sentence.


\section{10. Faulty sentence structure} 
\begin{itemize}
\item With so much going on in the world today is why it is so hard to 
keep up with everything. \ding{55}

\item With so much going on in the world, it can be hard to keep up. \ding{51}
\end{itemize}

\noindent When a sentence begins with a certain structure, then abruptly shifts to a different
one, it becomes disorderly and difficult to follow. 



\section{11. Unnecessary shift in verb tense} 
\begin{itemize}
\item She ran to the store and picks up some milk. \ding{55}

\item She ran to the store and picked up some milk. \ding{51}
\end{itemize}

\section{12. Lack of agreement between pronoun and antecedent}

\begin{itemize}
\item Each of the prisoners found happiness in their work. \ding{55}

\item Each of the prisoners found happiness in his work. \ding{51}
\medskip
\item Either Jeff or Robert will be required to give up their car. \ding{55}

\item Either Jeff or Robert will be required to give up his car. \ding{51}
\medskip
\item The campaign constantly changed its positions in the weeks before the election. \ding{51}

\item The campaign constantly changed their positions in the weeks before the election. \ding{51}
\end{itemize}

\noindent Pronouns and their antecedents must always agree in number. Three 
rules govern the choice between singular or plural pronouns. 1) Sentences that begin
with an indefinite pronoun (such as everyone and each) are \emph{always}
treated as singular. 2) If antecedents are joined by \emph{or} or \emph{nor}, 
the pronoun must agree with the \emph{closer} antecedent. 3) Collective nouns can be either 
singular or plural depending on whether the people are seen as a single unit or a 
group of individuals.

\section{13. Missing or misplaced possessive apostrophe} 

\begin{itemize}
\item The Baker Hill farm stand is proud to offer \hl{it's} vegetables for sale now. \ding{55}

\item The Baker Hill farm stand is proud to offer its vegetables for sale now. \ding{51}
\medskip
\item The Indian's best player is Ubaldo Jimenez. \ding{55}

\item The Indians' best player is Ubaldo Jimenez. \ding{51}
\end{itemize}

\section{14. It's / Its error} 


\begin{itemize}
\item Its unfair to make him pay for all the damages. \ding{55}

\item It's unfair to make him pay for all the damages. \ding{51}
\end{itemize}

\noindent\textbf{It's} is a contraction and means "it is" or "it has." \textbf{Its} is the
possessive form of it.

\section{15. Fused (run-on) sentence} 

\begin{itemize}
\item Jeff was Wisconsin's greatest dog trainer he could make a canine do virtually anything. \ding{55}

\item Jeff was Wisconsin's greatest dog trainer; he could make a canine do virtually anything \ding{51}
\end{itemize}

\noindent A fused sentence is also known as a "run-on" sentence. It occurs when two
clauses that could stand alone as complete sentences are placed together
without punctuation.


\section{16. Comma splice} 

\begin{itemize}
\item Indians once ruled the valley, they are all gone now. \ding{55}

\item Indians once ruled the valley, but they are all gone now. \ding{51}

\item Indians once ruled the valley; they are all gone now. \ding{51}
\end{itemize}

\noindent A comma splice occurs when two independent clauses are joined by a comma. To
revise, use a comma with a coordinating conjunction or a semicolon.



\section{17. Misplaced modifier}
\begin{itemize}
\item He wore his new coat to church, which was covered in cat hair. \ding{55}

\item He wore his new coat, which was covered in cat hair, to church. \ding{51}
\end{itemize}

\noindent The first example suggests that the church, rather than the coat, was covered in cat hair.




\section{18. Poorly integrated quotation} 

\begin{itemize}
\item Scholar Rod Andrews argues "I argue that there can be no 
reasonable discussions of Shakespeare's biography" (99). \ding{55}

\item Scholar Rod Andrews argues that "there can be no reasonable discussions 
of Shakespeare's biography" (99). \ding{51}
\end{itemize}

\noindent When you integrate borrowed material into your own writing, the 
"hybrid" sentence you create must satisfy grammar. Further, the quoted material should
blend seamlessly into your sentence structure.

\section{19. Unnecessary or missing hyphen} 
\begin{itemize}
\item He bought a nineteenth century painting. \ding{55}

\item He bought a nineteenth-century painting. \ding{51}

\item Boston has a lot of one way streets. \ding{55}

\item Boston has a lot of one-way streets. \ding{51}
\end{itemize}

\section{20. Sentence fragment}

\begin{itemize}
\item This school offers many classes. Such as Accounting and English. \ding{55}

\item This school offers many classes, such as Accounting and English. \ding{51}
\end{itemize}

\noindent A fragment is an incomplete thought. It is a dependent clause treated
as an independent clause.
