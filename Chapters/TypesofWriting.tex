
%-------------------------------------------------------------------------- %
 % TYPES OF WRITING AT THE COLLEGE LEVEL %--------------------------------------------------------------------------

\chapter{Types of College Writing}

Most students are trained in high school to write what is known as a
five-paragraph essay\textemdash a form of writing containing an introduction,
three paragraphs of support, and a summary conclusion. In that it encourages
students to think about introductions, structure, organization, paragraphing,
evidence, and reasoning, the five-paragraph essay is good training for novice
writers; however, this particular rhetorical form is \emph{extremely} limiting.
Only certain, rather paltry, thoughts and ideas may be placed within a five
paragraph structure. 

Your college coursework will require that you reach for rhetorical forms of an
entirely different nature. Little of the thinking you will be challenged to do
in college will fit within the constraints of the five-paragraph essay. Big
ideas, complex reasoning, and deep inquiry will require more sophisticated structures.
Rather than force all of your reasoning and inquiry into a preformulated and
supposedly universal structure, \emph{you will be asked to embrace the idea that
the formal properties of a piece of writing are determined by the particular needs of the
argument}. Thus, the shape of your reasoning will determine the essay's form. Of
course, every discipline has its own particular way(s) of writing; as you
progress through your coursework you will become more familiar with how your
chosen field of study presents its own kinds of academic discourse.

While academic writing takes on many forms, the following are the most common
modes of writing you will encounter. However, it is perhaps misleading to
present these various modes of academic writing as discrete things. Please
understand that there are no definite lines between these various kinds of academic
writing, and it is uncommon to write in one mode exclusively. The rhetorical tasks
you face in college will often require you to \emph{combine these modes in
various ways within a single piece of writing}. For example, an argument paper
will often involve synthesis and analysis, a research essay may use one or more
theories, and a theory paper might use close reading and analysis.

\hypertarget{argument}{} \section{Argument}

An argument paper requires you to make a claim about a debatable issue and then
defend that claim using evidence and reasoning. Virtually all academic writing
is argumentative in nature. Argumentative essays generally begin with an introduction
that explains the context for the argument and the specific issue, problem, or
question that the paper will address. Typically, the author of an argument will
use the end of the introduction to present a \textbf{thesis}\textemdash the main
idea or claim of the essay. However, this is not always the case. For example,
one argumentative form known as the "exploratory essay" replaces the thesis with a
question that is used to initiate an inquiry into an issue or problem. Typically, the
thesis appears near the end of this kind of essay, as the culmination of a process
of reasoning and inquiry.


\hypertarget{responseessay}{} \section{Response}

A response paper gives you free license to respond to a text without guidance.
Rather than a prompt or prescribed approach given to you by a professor, a
response paper allows you to engage a text on your own terms and write from your
own perspective.

While a response paper allows you to write about something you choose, your
effort should not be an impressionistic one where you only talk about your
personal feelings\textemdash what you like or dislike about the text. Rather,
you should seek to \emph{evaluate} and \emph{engage} the claims and ideas you
find in the text. Thus, a response is always argumentative in nature in that you will make 
claims and use reasoning and evidence as support.

Your response may seek to take issue with some of the thinking or reasoning put
forth in the reading. However, a good response doesn't just say "I agree with X"
or "I disagree with Y." Instead, \emph{explicit reasons are stated and
explanations are made that challenge or support the writer's ideas}. A good
response essay might alternatively attempt to forge a connection between two or
more texts by demonstrating a relationship between the ideas or arguments
involved\textemdash contrasting, comparing, and evaluating the claims or ideas
in the texts. For example, how might Author A respond to Author B? How
do their views compare? Can their views be reconciled? Is one view superior?

\section{Exposition} An expository essay is one in which you report on, define,
summarize, clarify and/or explain a concept, process, idea, or text. Expository
essays involve several key patterns such as compare and contrast, cause and
effect, problem and solution, or definition. The purpose of this kind of
writing is to provide information to an audience unfamiliar with the subject or
to demonstrate for a professor that you have understood course material.

\hypertarget{synthesisessay}{} \section{Synthesis}

As the name suggests, synthesis essays focus on \emph{combining} and
\emph{connecting}. Your focus in a synthesis essay is to explain to your
audience the ways in which two or more arguments or ideas relate to one another.

Students attempting synthesis for the first time often make the mistake of
organizing their essays by source. For example, this student might introduce two
authors in her introduction, summarize Author A, summarize Author B, then
conclude by noting the broad similarities and differences in the two authors'
thinking. This is \emph{not} synthesis.

In a synthesis essay you should try to organize your essay by \emph{topic} or
\emph{questions at issue} rather than by sources. Rather than try to summarize
the essays separately, a synthesis will attempt to discover the various things
that the authors discuss\textemdash the questions, ideas, and arguments they
have in common\textemdash then present those things in an organized and
meaningful way. Thus, your objective in a synthesis is to bring two or more
distinct sources into a relationship by explaining to your reader the various
ways in which the sources are in \emph{dialogue}.

To begin a synthesis, ask yourself the following questions about the readings
you plan to synthesize: \begin{quote} What are the positions, arguments, and
ideas that the source materials have in common? Are the authors all concerned
about the same problem(s)?  Are their arguments similar or do they differ? What
reasoning supports their arguments? Do they offer similar conclusions or are
there significant differences? \end{quote}

\noindent After answering these questions \emph{exhaustively}, write an essay
that examines the relationship between the various authors’ arguments, comparing
and contrasting their views.


Synthesis is very textual in nature: \emph{you must show explicit textual
evidence for each of the claims you attribute to the other authors}. Using
\hyperlink{summary}{\color{Ahrenge}{summary}},
\hyperlink{paraphrase}{\color{Ahrenge}{paraphrase}}, and
\hyperlink{quotation}{\color{Ahrenge}{quotation}}, compare and contrast the
authors’ positions. Make sure to cite each of these appropriately. Use clear
\hyperlink{signalphrase}{\color{Ahrenge}{signal phrases}} to transition between
your presentations of the various author’s ideas or works.




\hypertarget{closereadingessay}{} \section{Analysis/Close Reading} Analytical
writing involves paying close attention to particular elements of a thing and
how those discrete elements work together to produce a whole. While analysis
always involves breaking things down and meticulously examining the particulars,
the ultimate goal of any analysis is to explain what something means or how it
works. In the case of literary criticism, you might perform an analysis of a
poem and then attempt to explain its meaning to your audience.  Rather than
quote an outside authority, you will instead provide \emph{your} interpretation
of the text using only the words of the poem itself as evidence. This process is
often referred to as "close reading." While this process may be performed on a
poem or a scene in a novel, a close reading may also be made of a film
sequence, a piece of artwork, a photograph, a built structure, a tribal dance, or
even a new fashion trend.

In more scientific disciplines you might examine a collection of data, then
describe in detail how this information leads to an broader explanation, theory,
or conclusion. In all cases, the analysis you perform should be used to support
a strong thesis\textemdash an idea or that you want your audience to accept as
true.

\hypertarget{theoreticalessay}{} \section{Theoretical Writing} The theoretical
essay is one of the most common forms of academic writing. Using a theory is
like using a tool: you take it with you to your job of reading and interpreting
a text and use it to uncover ideas and shape thoughts. Sometimes people refer to
theoretical arguments as “lens” essays since you view the text(s) you are
analyzing through the theory you have chosen. Like a lens, the theory will color
the text, bring certain things into focus, and make others fade out of view.

We might, for example, use a feminist theory to examine a novel. In this case,
the theory would sensitize us to certain aspects of the text such as the power
relationships between the female and male characters or how social authorities
or institutions treat men and women differently. Alternatively, we might perform
a Marxist analysis which would cause us to study how social class and
disparities in wealth shape the narrative and the various characters' outcomes
in the fictional world they inhabit. But theory is not just for the analysis of
fiction. We might, for example, appropriate some economic, sociological, or
anthropological theory to analyze how people behave at the mall or use some
psychological model to explain how students behave in dormitories.

There is an extraordinary variety of extant theoretical models that may be used
for the interpretation of texts, cultural forms, and various kinds of data. In
fact, every field of study uses theory in some way\textemdash from literary
criticism to the hard sciences. As you begin to specialize within a chosen
discipline of study, you will encounter the theoretical models that are
important for that discipline.

\hypertarget{researchessay}{} \section{Research}

A \hyperlink{academicresearch}{\color{Ahrenge}{research}} paper requires you to
draw on outside sources in addition to your own thinking. Research writing often
includes many of the kinds of writing described above. Indeed, research writing
involves a coordination of all of the previously mentioned skills and rhetorical
modes. For example, your research paper may involve one or more theories,
perform various kinds of analysis, synthesize the thinking of many other writers
or thinkers, and make one or more arguments.

When you make an academic argument, you are often entering a conversation that
existed long before you appeared. When you write, you must remain mindful of the
conversations that came before and nestle your views within those that already
exist. In short, you must demonstrate that your ideas have a \emph{context}.
This is where \hyperlink{academicresearch}{\color{Ahrenge}{research}} comes into
play. Before you can responsibly offer your views, you must know what the
critical conversation is, what arguments are being made, and what questions are
important (or irrelevant) to the debate.

Properly done, \hyperlink{academicresearch}{\color{Ahrenge}{research}} ensures
that what you write is a true contribution to the ongoing discussion, and not a
pointless exercise in repetition. The whole point, after all, is to move the
scholarly conversation further down the road.
