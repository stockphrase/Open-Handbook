%-----------------------
% Paragraphs and Topic Sentences
%-----------------------
\hypertarget{organization}{}
\chapter{Paragraphs \& Topic Sentences}

\begin{quote}
\small
I can remember picking up my father's books before I could read. The words themselves were mostly foreign, but I still remember the exact moment when I first understood, with a sudden clarity, the purpose of a paragraph. I didn't have the vocabulary to say "paragraph," but I realized that a paragraph was a fence that held words. The words inside a paragraph worked together for a common purpose. They had some specific reason for being inside the same fence.

\textemdash Sherman Alexie, "\href{http://articles.latimes.com/1998/apr/19/books/bk-42979}{Superman and Me}."

\end{quote}

A paragraph is a unit of text devoted to developing an idea. The idea is usually stated 
in a \textbf{topic sentence}. The topic sentence most commonly appears as the paragraph's first sentence;
however, in what is known as a periodic paragraph, the topic sentence will appear at the end.

Paragraphs should be unified. Every sentence in the paragraph should focus on the main idea expressed in the topic sentence. Further, the paragraph's individual sentences should be presented in a logical order and flow naturally from one to the other. It may be helpful to think of the paragraphs as minature essays, each with their own thesis, development, and proof. 

\section{How can I make my paragraphs unified?}

\begin{itemize}

\item Identify the main idea of the paragraph then remove or revise any sentence that does not 
develop that main idea or repeats ideas offered in another sentence.

\item Consider the order of your sentences. Is there a reason why they are in the order 
they are, or do they need to be rearranged to make sense to a reader?

\item Think about employing some \textbf{transitions} at the beginning 
of some of the sentences. These will help you pinpoint the relationships between your 
ideas/sentences and thus clarify these relationships for your reader. To help guide your readers in understanding the relationships between the sentences of your paragraphs, you might use words like \emph{For example}, \emph{However}, \emph{Additionally}, \emph{Specifically}, \emph{On the other hand}, \emph{Obviously}, \emph{As a result}, \emph{In distinction to}, \emph{In other words}, \emph{Significantly}, etc.

\item Repeat key words to remind your audience of the paragraph's focus.

\end{itemize}
 
\section{Strong topic sentences}
Topic sentences function like a miniature thesis that communicates the purpose or main idea of a paragraph. It is important that your topic sentences are clear and accurately reflect the nature of the paragraph that follows it. 

Most commonly, topic sentences are strong, declarative statements that make a claim. The sentences that follow the topic sentence in the paragraph are used to support that claim. However, a topic sentence may also be a question. In this case, the sentences that follow the topic sentence are used to move toward a conclusion or further development of the question. 
 
\section{Example paragraphs}

This paragraph is part of a larger student essay performing a theoretical analysis of Wes Anderson's 1998 film \href{http://en.wikipedia.org/wiki/Rushmore_%28film%29}{\emph{Rushmore}}. The essay uses theoretical ideas borrowed from Brazilian educational theorist Paulo Freire. According the the author, the film's central character, Max Fischer, fails to live up to the standards for education that Freire articulates.

\subsection*{An incoherent paragraph:} 

\bigskip

\begin{tcolorbox}[enhanced,width=4.2in,left=.3in, right=.3in,
   drop fuzzy shadow southeast,
    boxrule=0.4pt,sharp corners,colframe=black!80!black,colback=white!10]

\medskip

{\small
\begin{doublespacing}

\hl{An incoherent paragraph:}\medskip

\hspace{.5cm}Liberation education culminates in an effort to change the world. However, according to Freire, this change must embrace a communitarian philosophy. This is what Max Fischer fails to understand. Max is keen to change the world, shaping it to his needs and wants, but he fails to understand Freire's imperatives on community, dialogue and consensus. The views, ideas, and values held by a community are used to change the world, not one individual's desire. Rather than shape the world \emph{with} others, Max insists on altering the world for himself alone.

\medskip

\end{doublespacing}}

\end{tcolorbox}

\hl{Analysis}: The \textbf{topic sentence} of the paragraph does not represent the true nature of the paragraph. The paragraph is about Max's failure to embrace a core idea expressed by Freire; however, the topic sentence suggests the paragraph will only discuss something called "liberation education." Further, the paragraph's sentences are not presented in a helpful or logical order.

\subsection*{Revised paragraph:}

\bigskip

\begin{tcolorbox}[enhanced,width=4.2in,left=.3in, right=.3in,
   drop fuzzy shadow southeast,
    boxrule=0.4pt,sharp corners,colframe=black!80!black,colback=white!10]

\medskip

{\small
\begin{doublespacing}

\hl{Revised paragraph:}\medskip

\hspace{.5cm}Although Freire argues that liberation education culminates in an effort to change the world, Max Fischer's efforts in this regard fail to embrace the philosophy of community that Freire demands. While Freire beckons us to become “transformers of [the] world” (73), he insists that it must only happen after a process of “dialogue”: an open exchange of ideas between equal partners (78-9). For Freire, these moments of co-inquiry are used to transform the world into a more democratic and free society (86). While Max is keen to change the world, he selfishly shapes it to \emph{his own needs and wants}, failing to understand Freire's insistence on community, dialogue, and consensus. Rather than shape the world \emph{with} others, as Freire recommends, Max insists on altering the world for himself alone.

\medskip

\end{doublespacing}}

\end{tcolorbox}

\hl{Analysis}: This revised paragraph uses the \textbf{topic sentence} to announce that the passage will focus on how Max Fischer fails to live up to a standard expressed by Freire. This new topic sentence reflects the nature of the paragraph much more faithfully. Furthermore, the author arranges the sentences so as to educate the audience about the Freire's theory before using it to criticize the actions of the film's main character, Max Fischer. The result is a more logical and intelligible paragraph.
