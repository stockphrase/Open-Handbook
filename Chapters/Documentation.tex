
%---------------------------------------------------------------------------
% DOCUMENTATION OF SOURCES
%------------------------------------------------
\hypertarget{citation}{}

\chapter{Documentation of Sources}

Accurately documenting sources is a vital aspect of any process of inquiry. If
you fail to  properly document your sources, your readers will be unable to
follow your research,  validate your claims, or judge the quality of your
argument. Furthermore, failing to  properly cite a source (whether summarized,
paraphrased, or quoted) opens you to the  charge of
\hyperlink{plagiarism}{\color{Ahrenge}{plagiarism}}, a serious academic offense.

Scholars avoid plagiarism and give credit to the thinking and writing of others
using a  variety of citation formats, or "styles." As you work to complete your
degree in college  you will encounter a number of these citation formats. In
fact, each discipline has a  preferred style. The humanities use MLA, psychology
uses APA, history and other social  sciences use Chicago. There are many others.
As you begin to specialize in a particular field of study, you will be required
to master the citation style used by your discipline.  This brief handbook,
however, will only introduce you to two of the most common styles: MLA and
Chicago.

Although citation formats differ significantly, they all have two primary
components:  \hloy{in-text citations} and a \hloy{final bibliography}. As
the name suggests, in-text  citations are used to reference the work of others
within the text itself; the  bibliography contains an ordered list of all the
in-text citations contained within a  piece of writing.

%-----------------------------------------------------------------------------
%END OF SECTION
%-----------------------------------------------------------------------------
