%------------------------------------------------------------------------------
% Altering Sources
%-----------------------------------------------------------------------------

\chapter{Altering Sources}

In your life as a writer you will frequently encounter situations where you must
alter a source in some way. Generally, these alterations are used to satisfy
grammar and/or make your writing easier to follow by removing or adding words or
phrases. The \emph{alteration or addition} of individual words in a source text
is made using \hloy{brackets}: [ ]. We use \hloy{ellipses}\textemdash a
series of periods\textemdash to show when we have \emph{removed} words or
sentences from a source.

These alterations should be used \emph{sparingly}. In fact, you should only
alter a source if there is no other practical way to include the material in a
quotation. Overuse of brackets or ellipsis will make your writing appear artless
and lazy. However, when altering a source proves an indispensable path to
crafting readable prose, brackets and ellipsis are helpful tools.

Be careful that the quote you present to the reader is an accurate reflection of
the original text; it is improper to alter a text so that it says something the
author didn’t really intend.

\hypertarget{ellipsis}{}
\section{Ellipsis [. . .]} We use ellipses to show when we have altered a source
text by \emph{omitting} words, sentences, or paragraphs. However, be careful
that the quote you present to the reader is an accurate reflection of the
original text.

\subsection{How do I use ellipsis?}

\begin{tcolorbox}[enhanced,width=4.2in,left=.3in, right=.3in,
   drop fuzzy shadow southeast,
    boxrule=0.4pt,sharp corners,colframe=black!80!black,colback=white!10]

\medskip

{\small
\begin{doublespacing}
\textbf{\hl{Source Text}}
\smallskip

\hspace{.5cm}At the heart of the environmentalist worldview is the conviction that human physical and spiritual health depends on sustaining the planet in a relatively unaltered state. Earth is our home in the full, genetic sense, where humanity and its ancestors existed for all the millions of years of their evolution. Natural ecosystems\textemdash forests, coral reefs, marine blue waters\textemdash maintain the world exactly as we would wish it to be maintained. Our body and our mind evolved precisely to live in this particular planetary environment and no other. When we debase the global environment and extinguish the variety of life, we are dismantling a support system that is too complex to understand, let alone replace, in the foreseeable future (238).

\bigskip

\noindent\textemdash E.O. Wilson, “\href{https://doi-org.dartmouth.idm.oclc.org/10.1016/0303-2647(93)90052-E}{Is Humanity Suicidal?}”

\bigskip

\end{doublespacing}}

\end{tcolorbox}

\begin{itemize}

\item \hloy{Omission in the middle of a sentence}: \begin{quote}Wilson argues
that “Natural ecosystems . . . maintain the world exactly as we would wish it to
be maintained” (50).\end{quote}


\item \hloy{Omission of the ending of one sentence and the beginning of
another:} \begin{quote}As Wilson states, “At the heart of the environmentalist
worldview is the conviction that human physical and spiritual health depends on
sustaining the planet . . . . where humanity and its ancestors existed for all
the millions of years of their evolution” (50).\end{quote}

\item \hloy{Omission of one or more sentences:}\begin{quote} Wilson says that
“At the heart of the environmentalist worldview is the conviction that human
physical and spiritual health depends on sustaining the planet in a relatively
unaltered state. . . . When we debase the global environment and extinguish the
variety of life, we are dismantling a support system that is too complex to
understand, let alone replace, in the foreseeable future” (50).\end{quote}

\end{itemize}

\section*{Special considerations with ellipsis}

\begin{itemize} \item\hloy{Omission at the beginning of a sentence:}

It is often unnecessary to use ellipsis when you have omitted the beginning of a
sentence. Since the portion of the quote you present will begin with a lower
case letter, it will be obvious to the reader that it does not begin the
sentence in the original text. Thus, for the most part, ellipsis will only occur
in the middle or the end of a sentence. Consider the following examples; only
example \textbf{c} requires the use of ellipsis to begin a quote:

\textbf{a) With bracket}:  \begin{quote}“[H]uman physical and spiritual health,”
Wilson writes, “depends on sustaining the planet in a relatively unaltered
state” (50).\end{quote}

\textbf{b) Lowercase letter:} \begin{quote}According to Wilson, “human physical
and spiritual health depends on sustaining the planet in a relatively unaltered
state” (50).\end{quote}

\textbf{c) Proper noun in the original:} \begin{quote}As Terrance Smith relates, “. . .
Wilson argues that we must protect the planet if we want to protect ourselves”
(99).\end{quote}

Since “Wilson” is a proper noun, it is capitalized. This may lead the reader to falsely assume that it is the beginning of a sentence in the original text. Therefore, ellipsis is used to clarify that the original sentence begins earlier.

\item \hloy{Using a short phrase or word:}

When quoting a short phrase or single word there is no need to use ellipsis
since it will be clear to your readers that this has been taken from a longer
sentence:

\begin{quote} Wilson describes Earth as “our home” (50).\end{quote}

\end{itemize}

\hypertarget{brackets}{}
\section{[Brackets]} \subsection{How do I use brackets?}

Square brackets are used to indicate an alteration or addition to a source text.
Generally, you will use brackets in three circumstances: 1) to add clarifying
information, 2) to alter a word in the interest of grammar, or 3) to alter
capitalization. A few examples:

\begin{enumerate}

\item James McMinnis maintains that “the city [Los Angeles] is one of the most
horrific places on the face of the earth” (88).

\item Dillard concludes her essay by saying that she “think[s] it would be well,
and proper, and obedient, and pure, to grasp your one necessity and not let it
go” (34).

\item “[H]uman physical and spiritual health,” Wilson writes, “depends on
sustaining the planet in a relatively unaltered state” (50). 

\end{enumerate}

%-----------------------------------------------------------------------------
% END OF SECTION
%-----------------------------------------------------------------------------
