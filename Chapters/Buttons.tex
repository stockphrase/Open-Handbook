

\chapter{Comment Buttons}
\hypertarget{commentbuttons}{}

\newcommand\invisiblesection[1]{%
\refstepcounter{section}%
\addcontentsline{toc}{section}{\protect\numberline{\thesection}#1}%
\sectionmark{#1}}

\iconsectioninToC  


There is
\href{http://www.inventio.us/ccc/1988/12/robert-j-connors-and-andrea-a.html}{strong
statistical evidence} that almost 100\% of student writing problems are limited
to 20 discrete errors. As a result, I find myself writing the same things over
and over (and over) at the expense of more substantive remarks. I developed
Comment Buttons™ as a way of speeding up this process.

Comment Buttons™ are a kind of shorthand\textemdash a system of symbols representing
specific errors that I quickly jot into the margins of student writing. Later,
you may “decode” the symbols with the documentation below. Pay close attention
to the codes you generate; they will help you become more aware of the specific
problems you have in your writing and will thus help you to eliminate them.

An additional benefit of Comment Buttons™ is that it requires you to take
ownership of the revision process. Rather than merely retyping what I have
written on the pages of your essay, you must thoughtfully examine the
grammatical rules and examples yourself, then make appropriate revisions to your
work.\\

\invisiblesection{Audience} \begin{center}

\resizebox{1.5cm}{1.5cm}{%
\begin{tikzpicture}[scale=4, transform shape]
\tikzstyle{every node} = [circle, circular drop shadow, fill=Ahrenge!80]
\node (a) at (0, 0) {A};
	
\end{tikzpicture}
}

\bigskip

{\huge 1. Audience} \end{center}

It is important that you carefully consider the intended audience of your
writing. If your essay makes use of other texts, theories, or ideas, it is
important that you properly introduce them and provide appropriate context. As
you write, pretend that your audience is someone who is not taking part in our
class. What context do you need to provide your audience with so that your
ideas/thesis will make sense? For more advice on this subject,
visit the \hyperlink{audience}{\color{Ahrenge}{chapter on audience}} in this handbook.

\invisiblesection{Wow!} \begin{center}
\resizebox{1.5cm}{1.5cm}{%
\begin{tikzpicture}[scale=4, transform shape]
\tikzstyle{every node} = [circle, circular drop shadow, fill=Ahrenge!80]
\node (a) at (0, 0) {!};
	
\end{tikzpicture}
}

\bigskip


{\huge 2. Wow!} \end{center}

This button indicates that you have made an unusually sharp and penetrating
thought, idea, or claim. It might also indicate a fantastic turn of phrase or
beautiful and fluid moment of expression. Since this idea, observation, or
argument was so wonderful, you might consider whether you should expand on it a
bit, or otherwise make it a more central feature of your essay.

\invisiblesection{Uh, What?} \begin{center}
\resizebox{1.5cm}{1.5cm}{%
\begin{tikzpicture}[scale=3, transform shape]
\tikzstyle{every node} = [circle, circular drop shadow, fill=Ahrenge!80]
\node (a) at (0, 0) {?};
	
\end{tikzpicture}
}

\bigskip


{\huge 3. Uh, What?} \end{center}

This button refers to a statement that is confusing, ambiguous, or awkwardly
phrased. If reading the passage out loud to yourself does not reveal the problem
to you, please come by and see me for an explanation.

\invisiblesection{Add Space} \begin{center}
\resizebox{1.5cm}{1.5cm}{%
\begin{tikzpicture}[scale=4, transform shape]
\tikzstyle{every node} = [circle, circular drop shadow, fill=Ahrenge!80]
\node (a) at (0, 0) {\#};
	
\end{tikzpicture}
}
\bigskip

{\huge 4. Add Space} \end{center}

This symbol is a common copywriter’s mark that simply means to \emph{add a
space} between two letters or some punctuation. For example, if you write
“everyday,” as one word but mean “every day,” as two words, I will place this
mark in the margin and draw a line to the place where the space should be added.

\invisiblesection{Colloquial Language} \begin{center}
\resizebox{1.5cm}{1.5cm}{%
\begin{tikzpicture}[scale=3, transform shape]
\tikzstyle{every node} = [circle, circular drop shadow, fill=Ahrenge!80]
\node (a) at (0, 0) {CL};
	
\end{tikzpicture}
}

\bigskip


{\huge 5. Colloquial Language} \end{center}

Colloquial language is a fancy term that means common or everyday speech. Slang,
for example, is considered colloquial language. You should avoid common speech
or slang in your formal writing assignments. You want your audience to see you
as a scholar who speaks with authority and knowledge. For example, words like
“shady” or “ginormous” are better rendered “conspicuous” and “commodious.”

\invisiblesection{Close Space} \begin{center}
\resizebox{1.5cm}{1.5cm}{%
\begin{turn}{90}
\begin{tikzpicture}[scale=3, transform shape]
\tikzstyle{every node} = [circle, circular drop shadow, fill=Ahrenge!80]
\node (a) at (0, 0) {( )};
	
\end{tikzpicture}
\end{turn}
}
\bigskip


{\huge 6. Close Space} \end{center}

This indicates that you should close the space between two words or punctuation
marks. For example, I may draw this symbol when you write “every day” as two
words but meant to use the word “everyday,” as one word.

\invisiblesection{Citation Error} \begin{center}
\resizebox{1.5cm}{1.5cm}{%
\begin{tikzpicture}[scale=3, transform shape]
\tikzstyle{every node} = [circle, circular drop shadow, fill=Ahrenge!80]
\node (a) at (0, 0) {C};
	
\end{tikzpicture}
}
\bigskip

{\huge 7. Citation} \end{center}

This button indicates that there is an error in your citation. Either there is a
need for a citation or the citation is improperly formatted. If you are confused
about citation, investigate the matter in the chapters on citation and working
with sources (or ask me for an explanation).

\invisiblesection{ESL} \begin{center}
\resizebox{1.5cm}{1.5cm}{%
\begin{tikzpicture}[scale=3, transform shape]
\tikzstyle{every node} = [circle, circular drop shadow, fill=Ahrenge!80]
\node (a) at (0, 0) {ESL};
	
\end{tikzpicture}
}

\bigskip

{\huge 8. ESL} \end{center}

Some students are actually \emph{far more advanced} than their American
counterparts: they are taking this class in their \emph{second} language. This
button indicates a place where you have used non-standard English or have made a
common mistake for ESL writers.

\invisiblesection{Capitalize} \begin{center}
\resizebox{1.5cm}{1.5cm}{%
\begin{tikzpicture}[scale=4, transform shape]
\tikzstyle{every node} = [circle, circular drop shadow, fill=Ahrenge!80]
\node (a) at (0, 0) {\underline{\underline{a}}};

\end{tikzpicture}
}
\bigskip

{\huge 9. Capitalize} \end{center}

Generally, this is just an error in proofreading. This button means to
capitalize the underlined letter.

\invisiblesection{Discuss with Me} \begin{center}
\resizebox{1.5cm}{1.5cm}{%
\begin{tikzpicture}[scale=4, transform shape]
\tikzstyle{every node} = [circle, circular drop shadow, fill=Ahrenge!80]
\node (a) at (0, 0) {D};

\end{tikzpicture}
}

\bigskip

{\huge 10. Discuss with Me} \end{center}

Some things are too complicated to be placed in the margins of your essay. This
button merely invites you to discuss the issue with me after class or during an
office hour. I am happy to explain things in more detail.

\invisiblesection{Grammar Error} \begin{center}
\resizebox{1.5cm}{1.5cm}{%
\begin{tikzpicture}[scale=3.5, transform shape]
\tikzstyle{every node} = [circle, circular drop shadow, fill=Ahrenge!80]
\node (a) at (0, 0) {G\textsubscript{\tiny 2}};

\end{tikzpicture}
}

\bigskip

{\huge 11. Grammar Error} \end{center}

This button indicates that you have made a grammatical error of some sort. If
there is a number included in the button, it corresponds to the list of the
\hyperlink{Top20}{\color{Ahrenge}{Twenty Most Common Errors}} in the previous
chapter. Consult the previous chapter for a description of the error and revise
appropriately: 1) Missing comma after an introductory element; 2) Wrong word; 3)
Incomplete or missing documentation; 4) Vague pronoun reference; 5) Spelling
error (including homonyms); 6) Faulty parallelism; 7) Unnecessary comma; 8)
Missing comma with a nonrestrictive element; 9) Missing comma in compound
sentence; 10) Faulty sentence structure; 11) Unnecessary shift in verb tense;
12) Lack of agreement between pronoun and antecedent; 13) Missing or misplaced
possessive apostrophe; 14) Its/It's error; 15) Fused (run-on) sentence; 16)
Comma splice; 17) Misplaced modifier; 18) Poorly integrated quotation; 19)
Unnecessary or missing hyphen; 20) Sentence fragment.


\bigskip

\invisiblesection{Strike} \begin{center}
\resizebox{1.5cm}{1.5cm}{%
\begin{tikzpicture}[scale=4,transform shape]
\tikzstyle{every node} = [circle, circular drop shadow, fill=Ahrenge!80]
\node (a) at (0, 0) {\cancel {B}};
	
\end{tikzpicture}
}
\bigskip

{\huge 12. Strike} \end{center}

This button has two meanings: 1) It may mean to remove the textual element that
is marked out—this might be a letter, word, or punctuation mark. 2) However, it
may also be used to suggest a change from an uppercase to lowercase letter.

\invisiblesection{Interpretation Problem} \begin{center}
\resizebox{1.5cm}{1.5cm}{%
\begin{tikzpicture}[scale=4, transform shape]
\tikzstyle{every node} = [circle, circular drop shadow, fill=Ahrenge!80]
\node (a) at (0, 0) {I};
	
\end{tikzpicture}
}
\bigskip


{\huge 13. Interpretation} \end{center}

This indicates that you have made an error in the interpretation of the source
material. For example, you might claim that an author argues for X when in fact
he or she argues the opposite.

\begin{center} 
\resizebox{1.5cm}{1.5cm}{%
\begin{tikzpicture}[scale=4, transform shape]
\tikzstyle{every node} = [circle, circular drop shadow, fill=Ahrenge!80]
\node (a) at (0, 0) {L};
	
\end{tikzpicture}
}
\bigskip

\invisiblesection{Logic} {\huge 14. Logic} \end{center}

This button indicates that there is a logical inconsistency in your writing.
Most often this involves a “leap” in logic or an unexpected or disruptive turn
in your writing that creates confusion. This might also involve an argumentative
statement that you make that seems to contradict an earlier one. At worst, it
might involve a line of reasoning or a statement that contradicts your thesis
statement.

\invisiblesection{Punctuation Error} \begin{center}
\resizebox{1.5cm}{1.5cm}{%
\begin{tikzpicture}[scale=3, transform shape]
\tikzstyle{every node} = [circle, circular drop shadow, fill=Ahrenge!80]
\node (a) at (0, 0) {P};
	
\end{tikzpicture}
}
\bigskip

{\huge 15. Punctuation Error} \end{center}

This button indicates that you have either misused punctuation or have left some
needful punctuation out. Consult your handbook for help with the punctuation
error or ask me for help.

\invisiblesection{Quotation Error} \begin{center}
\resizebox{1.5cm}{1.5cm}{%
\begin{tikzpicture}[scale=3, transform shape]
\tikzstyle{every node} = [circle, circular drop shadow, fill=Ahrenge!80]
\node (a) at (0, 0) {{\fixproblemfont Q}\textsubscript{\tiny 2}};

\end{tikzpicture}
}
\bigskip

{\huge 16. Quotation Error} \end{center}

This button indicates a problem with the integration of quotations. Like the
grammar button above, quotation errors will be identified with the following
numbers: 1) \textbf{Grammar}: The integration of the quotation creates a
grammatical error in the sentence of which it is a part. 2) \textbf{Logic}: The
quote has no clear relationship to the argument you are attempting to make or
illustrate. 3) \textbf{No introduction}: The quote is not introduced. Use a
signal phrase or otherwise weave the quote into your own writing. Avoid merely
plopping stand-alone quotations into your own writing. 4) \textbf{No
explanation}: Readers want to know \emph{why} you've quoted another writer's
words. What is significant about this quote? How is it relevant to your thesis
or overall project? 5) \textbf{Error with alteration}: There is a problem with
ellipsis [. . .] or [brackets] to indicate an omission or addition. 6)
\textbf{Lack of flow}: The quotation does not flow with the text that appears
before it, creating a sudden shift in voice, emphasis, topic, or tone. 7)
\textbf{No justification}: Quotes must be justified. If the quote you use
contains only factual information or uses language that is ordinary, there is no
need to quote it. Instead, put the idea or facts into your own words and then
cite the author.

\invisiblesection{Repetitive} \begin{center}
\resizebox{1.5cm}{1.5cm}{%
\begin{tikzpicture}[scale=3, transform shape]
\tikzstyle{every node} = [circle, circular drop shadow, fill=Ahrenge!80]
\node (a) at (0, 0) {R};
	
\end{tikzpicture}
}
\bigskip

{\huge 17. Repetitive} \end{center}

This button indicates a repeated word, concept, or idea. It might also involve
something that you say over and over again. Also, you might do something
redundant, like reiterate the same thing a few times using only slightly
different language.

\invisiblesection{Subject/Verb Agreement} \begin{center}
\resizebox{1.5cm}{1.5cm}{%
\begin{tikzpicture}[scale=4, transform shape]
\tikzstyle{every node} = [circle, circular drop shadow, fill=Ahrenge!80]
\node (a) at (0, 0) {S/V};

\end{tikzpicture}
}
\bigskip

{\huge 18. Subject/Verb Agreement} \end{center}

This indicates that your subject and verb do not agree in number. Both must be
singular or both must be plural.

\invisiblesection{Topic Sentence} \begin{center}
\resizebox{1.5cm}{1.5cm}{%
\begin{tikzpicture}[scale=3, transform shape]
\tikzstyle{every node} = [circle, circular drop shadow, fill=Ahrenge!80]
\node (a) at (0, 0) {TS};

\end{tikzpicture}
}
\bigskip

{\huge 19. Topic Sentence} \end{center}

Topic sentences are very important in academic writing. A topic sentence
functions as a “mini thesis” that announces the subject of a paragraph. As you
revise your writing, ensure that you have strong, descriptive topic sentences
and that the paragraphs that follow are unified under that topic or idea. For
those of you who have difficulties with organization, revising topic sentences
and checking paragraphs for unity is the fastest route to improvement.

\invisiblesection{Transitions} \begin{center}
\resizebox{1.5cm}{1.5cm}{%
\begin{tikzpicture}[scale=3, transform shape]
\tikzstyle{every node} = [circle, circular drop shadow, fill=Ahrenge!80]
\node (a) at (0, 0) {T};
	
\end{tikzpicture}
}
\bigskip

{\huge 20. Transitions} \end{center}

This button indicates a jarring or unexpected transition or sudden “leap” into
another subject. This often occurs between paragraphs, but may just as easily
occur between sentences if you suddenly change topic or emphasis. Sudden shifts
of focus such as this carry a very negative rhetorical consequence: it makes you
appear careless or illogical to your readers.

\invisiblesection{Spelling} \begin{center}
\resizebox{1.5cm}{1.5cm}{%
\begin{tikzpicture}[scale=3, transform shape]
\tikzstyle{every node} = [circle, circular drop shadow, fill=Ahrenge!80]
\node (a) at (0, 0) {S};
	
\end{tikzpicture}
}
\bigskip

{\huge 21. Spelling} \end{center}

This button indicates misspelled word. It may also include a "wrong word error"
where a word is spelled correctly, but is not the intended word. For example, "I
will \hl{meat} you after school." Sounds dreadful!

\invisiblesection{Word Choice} \begin{center}
\resizebox{1.5cm}{1.5cm}{%
\begin{tikzpicture}[scale=3, transform shape]
\tikzstyle{every node} = [circle, circular drop shadow, fill=Ahrenge!80]
\node (a) at (0, 0) {WC};
\end{tikzpicture}
}
\bigskip

{\huge 22. Word Choice} \end{center}

This indicates a word choice error. Commonly, the word in question has a
definition that is inconsistent with the meaning you intend or is otherwise
counterproductive to the meaning you intend. The best thing to do here is to
look up the word in a dictionary to examine its meaning(s). Additionally, I may also use this symbol if the word in question may create a
problem for your audience or argument. For example, calling someone “brain-dead”
or a “midget” might offend your readers. This might also involve the lack of
inclusive, \href{http://www.ncte.org/positions/statements/genderfairuseoflang}{gender-fair language}. An example of this would be the overuse of the
pronoun “he” to refer to a generic person or the use of “man” to signify both
women and men.

\invisiblesection{Unified Paragraph} \begin{center}
\resizebox{1.5cm}{1.5cm}{%
\begin{tikzpicture}[scale=3, transform shape]
\tikzstyle{every node} = [circle, circular drop shadow, fill=Ahrenge!80]
\node (a) at (0, 0) {UP};
	
\end{tikzpicture}
}
\bigskip

{\huge 23. Unified Paragraph} \end{center}

This indicates that your paragraph is not unified under a single topic. You
should have a clear topic sentence and what follows in the paragraph should be
focused on that one idea. Generally, this button appears when paragraphs contain
multiple ideas or shifts in focus.

\invisiblesection{Italicize} \begin{center}
\resizebox{1.5cm}{1.5cm}{%
\begin{tikzpicture}[scale=3, transform shape]
\tikzstyle{every node} = [circle, circular drop shadow, fill=Ahrenge!80]
\node (a) at (0, 0) {/  /};
	
\end{tikzpicture}
}
\bigskip

{\huge 24. Italicize} \end{center}

The forward slashes on either side of a word indicates that it should be
\emph{italicized}. For example, you may have forgotten to italicize the title of
a book or the name of a journal as required by your citation style.
Alternatively, italics may also be used for emphasis, which asks your readers to
give more weight to a particular word in your writing. While this can be very
effective, it should be used \emph{sparingly}.

\newpage

\invisiblesection{New Paragraph} \begin{center}
\resizebox{1.5cm}{1.5cm}{%
\begin{tikzpicture}[scale=4, transform shape]
\tikzstyle{every node} = [circle, circular drop shadow, fill=Ahrenge!80]
\node (a) at (0, 0) {\fixproblemfont\P~};
\end{tikzpicture}
}
\bigskip


{\huge 25. New Paragraph} \end{center}

This symbol, known as a "pilcrow," means to begin a new paragraph. This
indicates a moment in your writing when it appears that you have begun a new
idea or changed focus such that a new paragraph is warranted.

\invisiblesection{Insert} \begin{center}
\resizebox{1.5cm}{1.5cm}{%
\begin{tikzpicture}[scale=4, transform shape]
\tikzstyle{every node} = [circle, circular drop shadow, fill=Ahrenge!80]
\node (a) at (0, 0) {\textasciicircum};
\end{tikzpicture}
}
\bigskip

{\huge 26. Insert} \end{center}

This symbol, known as a "caret," indicates that I have added a word or some
language to one of your sentences. This addition should be considered, not
automatically adopted. This is \emph{your} essay, after all.

\invisiblesection{Problem Area} \begin{center}

\resizebox{1.5cm}{1.5cm}{%
\begin{tikzpicture}
\fill[fill=Ahrenge!80, circular drop shadow] (0,0) circle (1cm);

\draw[line width=1pt, color=black] plot [domain=-.85:.85, samples=30, smooth] (\x,{0.1*rand});
\end{tikzpicture}
}
\bigskip

{\huge 27. Problem Area} \end{center}

This squiggly line will appear beneath the specific text that contains a
grammatical error, confusing statement, formatting concern, or other type of
problem. Look in the margin next to these marks for an explanation or comment.

\stdsectioninToC 
