%----------------------------------------------------------------------------------------
%	How to be a student
%----------------------------------------------------------------------------------------

\chapter{Professor Operator's Manual} 

I always find it is helpful when I arrive a new place that is governed by different, unwritten rules to seek out a friendly person and find out how things work. I’ll be your friendly person. One of the most difficult things to navigate at college is how to relate to your instructors. Here are a few marginally helpful things that may help you keep a good relationship with your teachers.

\section{Tactical errors with professors}

\begin{itemize}

\item \hloy{Timing is everything}. Asking your professor to meet and discuss your idea for a paper shortly before it is due reveals that you haven't started yet. There is a certain cutoff point after which you should not reveal such things to the person who will grade your work. While it is quite possible for a good student to write something amazing overnight, revealing this to your professor may not be to your advantage. 

\item \hloy{Did I miss anything important}? Never ask your professor if you "missed anything important" on a day you were absent. Yes, yes you did. It is all important. And even if it's not, pretend that it is.

\item \hloy{There are no dumb questions}. It has been said that there are no dumb questions. But this is wrong. There is an amazing variety of dumb questions asked with a great deal of frequency. One you can easily avoid is asking something that is already thoughtfully covered in the syllabus. Ensure the question you have about the class isn't answered in the syllabus before you ask your instructor. Of course, if something in the syllabus is confusing or ambiguous, please ask for clarification. 

\section{Asking for extensions}

\item \hloy{Extensions are only occasionally justified}. Sometimes life intervenes in our plans and rudely sets us back. If you get hit by a bus or contract ebola, asking for a paper extension is a reasonable request. But generally it is a bad idea to request an extension unless you find yourself in a similarly extreme situation. Many faculty are reluctant to give extensions because of a duty to the principle of fairness: if one student is granted an extension, the other students who turned their work in on time are disadvantaged. Students have been known to turn in several large projects on the same day without requiring extra time. These students accomplish this feat by planning ahead, usually by mapping out all of their assignments for an entire term in a calendar, then starting early. And, for what it's worth, I know of at least one professor who, while granting extra time on occasion, always grades the resulting work much more stringently than if it had been turned in on time. 

%\item \emph{I'm too busy}. Asking for an extension because you are "too busy" with other classes or have other work due at the same time is not a good look for you. Your professor might conclude that you consider those other courses more important than hers. She might conclude that because that is exactly what you are, in effect, saying. 

\section{Office hour \emph{faux pas}}

\item \hloy{Avoid showing up at your professor's office when it is not office hours}. Professors are very busy with research, administrative duties, and class preparation. Many also have family obligations. If your professor is in her office and it isn't office hours, she isn't waiting for you. You can, of course, set up an appointment to meet with your instructor at a more convenient time. 

\item \hloy{ I know you're in there!} I can't believe I have to say this, but the student who shows up at a professor's office at an unscheduled time, discovers a locked door, then \emph{proceeds to knock on the door until the professor hiding inside opens it,} has just performed a great feat of self-immolation.

\item \hloy{Don't bogart your professor}. If a line of students form during an office hour and you have been interacting with the professor for quite some time, allow your fellow students to enter. You can always return to the line or ask for a separate appointment to continue the discussion. 

\section{Honorifics and such}

\item \hloy{What do you call your teachers}? Almost every year a student will ask me "What do we call you?" on the first day of classes. This is hard to answer. All of your teachers at Dartmouth have terminal degrees in at least one academic field. Usually this means a Ph.D. They publish books and articles. They are asked to give speeches and talk on television. For some faculty, this has become so related to their own sense of self-worth that they will visibly cringe if you call them "Bill" or, even worse, "Mr. Smith." Let's call this group the \textbf{Prouds}. A second group, although just as accomplished as the Prouds, are less dependent on their academic accomplishments to gain psychological uplift. They would actually \emph{prefer} if you called them by their first name in order to better signal a desire to close the distance between teacher and pupil. Let's call them \textbf{Brofessors}. A third group, perhaps the most interesting of the bunch, are the \textbf{Pseudo-Brofs}. While they will look you directly in the eye and say "I don't care what you call me," or "call me Phyllis," in truth (for psychological reasons too complex to go into here) they crave the honorific "doctor" or "professor" even more than the Prouds. Now, students, what does this mean for you? It means that unless you love to court danger and/or have too much integrity to submit to authoritarian systems, the safe move is probably to call all your teachers "professor X" or "doctor X."


\section{Seeking help}

\item \hloy{Grade grubbing}. Your professors want to help you. Really, they do. \emph{But at some point, seeking help becomes a form of dependence}. Taken to extremes, your help-seeking becomes an avoidance of your own responsibility as a writer, a thinker, a student, a human. Over the years I have had many students who bring their work to me excessively and who want my approval on every edit of their written work. Obviously, the students who do this just want to do well, but consider what they lose in the bargain. At some point, the essay or project ceases to be the work of the student, as the line between authorship and critique first blurs and then becomes indistinguishable. The student has completely surrendered his right to think and know and argue \emph{on his own terms}, blithely giving away his freedom to an authority figure in the person of his teacher. College should be a series of experiences through which you \emph{gain ownership over your own thinking and define for yourself your ideas and values and meanings}. While your teachers are here to provide feedback, critique, and assistance, you should do your own work, think your own thoughts, write your own papers, and, through those experiences, become who you are.  

\section{Recommendation letters}

\item \hloy{Recommendation letters}. I have never turned down a student for a recommendation letter. I enjoy recommending my students to graduate programs, medical schools, and scholarship committees. Feel free to ask me, or any professor who knows you, for a letter. Just make sure that you ask early. Students who ask for letters with 12 hours notice do not usually get the best letters. It unnecessary to buy me coffee or lunch to secure one.

\end{itemize}
