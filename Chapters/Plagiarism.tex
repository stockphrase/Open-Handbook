
%----------------------------------------------------------------------------------------
% Plagiarism
%----------------------------------------------------------------------------------------


\chapter{Plagiarism}

\begin{quote} \small Perish those who said our good things before we did.

\textemdash Aelius Donatus \end{quote}


\hypertarget{plagiarism}{}
\section{Definition of plagiarism}

In the \href{http://www.dartmouth.edu/judicialaffairs/honor/index.html}{Academic Honor Principle}, Dartmouth College defines plagiarism as
\begin{quote}

the submission or presentation of work, in any form, that is not a student's own, without acknowledgment of the source. With specific regard to papers, a simple rule dictates when it is necessary to acknowledge sources. If a student obtains information or ideas from an outside source, that source must be acknowledged. Another rule to follow is that any direct quotation must be placed in quotation marks, and the source immediately cited. Students are responsible for the information concerning plagiarism found in \href{http://writing-speech.dartmouth.edu/learning/materials/sources-and-citations-dartmouth}{\emph{Sources and Citations at Dartmouth College}}.
\end{quote}

\noindent Consult the \href{http://www.dartmouth.edu/judicialaffairs/honor/index.html}{Academic Honor Principle} for more information on academic dishonesty and a description of its severe consequences.

\section{Examples of plagiarism}

While plagiarism appears in many forms and degrees, we may broadly classify them in three categories: 1) the wholesale copying of another's essay or project, 2) the adoption of certain phrases or words from another text without proper attribution (also known as "patchwriting,") and 3) the paraphrase of another's writing without proper attribution or citation. 

Below you will find examples of these three categories of plagiarism. Each example plagiarizes a passage taken from Patricia Nelson Limerick's book, \emph{The Legacy of Conquest}:\bigskip

\begin{tcolorbox}[enhanced,width=4.2in,left=.4in, right=.4in,
   drop fuzzy shadow southeast,
    boxrule=0.4pt,sharp corners,colframe=black!80!black,colback=white!10]

\smallskip

\begin{flushright}

%\rule{4cm}{.7pt}

{\economica \emph{The Legacy of Conquest} \hspace{6pt} 35 }

\end{flushright}

{\small
\begin{doublespacing}

\hspace{.5cm}When academic territories were parceled out in the early twentieth century,
anthropology got the tellers of tales and history got the keepers of written
records. As anthropology and history diverged, human differences that
hinged on literacy assumed an undeserved significance. Working with oral,
preindustrial, prestate societies, anthropologists acknowledged the power
of culture and of a received worldview; they knew that the folk conception
of the world was not narrowly tied to proof and evidence.  But with the
disciplinary boundary overdrawn, it was easy for historians to assume that
literacy, the modern state, and the commercial world had produced a different
sort of creature entirely\textemdash humans less inclined to put myth over reality,
more inclined to measure their beliefs by the standard of accuracy and
practicality. (35)

\bigskip

\noindent\textemdash Patricia Nelson Limerick, \href{http://libcat.dartmouth.edu/record=b1422593~S1}{\emph{The Legacy of Conquest}}.

\bigskip

\end{doublespacing}}

\end{tcolorbox}


\subsection{Word-for-word copying}

Word-for-word copying, such as the following example, is completely unacceptable whether it occurs by mistake or design. Care must be taken when taking notes and typing quotations to avoid representing the words of other authors as your own:

\begin{quote}
As we all know, \hl{when academic territories were parceled out in the early twentieth century, anthropology got the tellers of tales and history got the keepers of written records.} This made historians \hl{assume that literacy, the modern state, and the commercial world had produced a different sort of creature entirely\textemdash humans less inclined to put myth over reality, more inclined to measure their beliefs by the standard of accuracy and practicality.} 

\end{quote} 

\hypertarget{patchwriting}{}

\subsection{Patchwriting}

This form of plagiarism is often the result of sloppiness in the note-taking or drafting process. Using patches of a source text is perfectly reasonable; however, make sure that quotation marks are used and citations are given:

\begin{quote}

When the \hl{academic territories were parceled out in the early 1900s}, the disciplines \hl{diverged}. This made the differences that human beings had with regard to literacy \hl{assume an undeserved significance}. By \hl{overdrawing this disciplinary boundary}, historians began to believe that the subjects they studied were \hl{less inclined to put myth over reality} and more likely \hl{to measure their beliefs} through the excellent standards of \hl{accuracy and practicality} (Limerick 35). 
\end{quote}



\subsection{Paraphrase without attribution}

The following paraphrase would be perfectly acceptable were it to have a \hyperlink{signalphrase}{\color{Ahrenge}{signal phrase}} and a citation indicating that the ideas were taken from another author's work. Even though no language was taken directly from the source text, the ideas must be attributed to the author from whom they were borrowed. As it stands now, the author is declaring that she came up with these ideas and arguments herself:

\begin{quote}

During the early part of the last century the disciplines of anthropology and history separated. While anthropology focused on unlettered and illiterate communities, history became the study of societies who produced texts and records. Within the field of anthropology, a firm belief developed that oral cultures were characterized by mythological worldviews and superstitious beliefs; on the other hand, historians assumed that literate cultures were filled with individuals who only used reason and evidence to guide their thinking.
\end{quote}

