%----------------------- % Audience %-----------------------
\hypertarget{audience}{}

\chapter{Audience}

\begin{quote} \small [A] work of rhetoric is pragmatic; it comes into existence
for the sake of something beyond itself; it functions ultimately to produce
action or change in the world; it performs some task. In short, rhetoric is a
mode of altering reality, not by the direct application of energy to objects,
but by the creation of discourse which changes reality through the mediation of
thought and action. The rhetor alters reality by bringing into existence a
discourse of such a character that the audience, in thought and action, is so
engaged that it becomes mediator of change (3-4).

\textemdash Lloyd F. Bitzer, "\href{http://www.jstor.org/stable/40593346}{The
Rhetorical Situation}." \end{quote}

\section{Constructing the Audience}

Everything we write has an audience\textemdash the person or people we address
with our words. Even a private journal is addressed to a future version of the
writer's current self. To a large degree these audiences will determine what we say and how we say it.

It is easy to jump too quickly to the immediate purpose of our
writing\textemdash the idea we want to articulate or the viewpoint we hope to
convince others to adopt. In doing so we forget that \emph{how} we say
something is as important as \emph{what} we say, particularly when we address
people who don't share our values, culture, or life experiences. The presentation of our
arguments\textemdash the kinds of evidence we use to support it, the words we
choose to articulate it\textemdash must be tailored for our audience if we hope
to be successful.

The nature of audiences is often elusive and complex. We may never completely understand the character and motives of our audience members; this imperfect knowledge presents a great challenge for writing arguments. Your audience members may hold views or beliefs that, while quite opaque, greatly determine receptiveness to your message. And in some cases your audience may be completely unknown to you\textemdash for example, if you write for the web. Thus, imagining the audience for your message is often not an easy task; it is something that you will have to make a judgment about using whatever evidence you happen to possess at the time. 

Before you write anything, carefully analyze those people whom you desire to
persuade. To the best of your knowledge, take an inventory of what you know about the audience (or audiences)
you plan to address in your writing. This analysis should give you 
insight into how best to present your thinking, reasoning, and evidence. You might begin such an audience analysis by asking questions such as these: \begin{itemize}

\item Who is your audience or audiences? 
\item What are their values? 
\item What educational background(s) do they have? 
\item What political views do they hold?
\item What ideas or commonalities do you share with them? 
\item What does your audience already know about the topic you plan to present? 
\item What form will your audience expect that your writing will take? (For example, if the writing occurs within a specific academic discipline, your writing will need to adopt to the \hyperlink{citation}{\color{Ahrenge}{preferred style}} for that discipline: MLA, Chicago, APA, etc.)

\end{itemize}
\noindent Questions like these can help us imagine the audience we address in
our writing and gain a sense of the rhetorical situation we face. This kind of
intelligence will help us make good decisions about many aspects of the writing
process such as organization, diction, style, and evidence.

\section{Persuading the Audience} 

While there are many types of writing, the kind you will do in college is largely concerned with \emph{argumentation} and \emph{persuasion}; it is a form of reasoned discourse designed to change the audience's mind or cause them to adopt some new idea or plan of action. As you analyze your audience with these and other similar questions, imagine how these particular people will respond to the argument(s) you plan to present to them. For example:

\begin{itemize}

\item What sorts of constraints do you envision in getting your audience to accept your
argument? 
\item If your intended audience already has known positions you
oppose, how can you work carefully to convince them that their views should
change? 
\item What sorts of things should you avoid presenting in your message?
\item What common values or beliefs can be used to make your views more
appealing and consistent with your audience's outlook? 
\item How can you establish rapport with your audience, based on what you know? 
\item How can you demonstrate that you are an authority on the issue or problem at hand?

\end{itemize}

\noindent Done properly, an audience analysis will help you craft your argument
more effectively, adopt a proper tone, use appropriate vocabulary, and avoid
any rhetorical missteps that may alienate your readers.

 \section{Addressing a broad audience}

In college your audience will most often be your professor and fellow class
members. However, when you write you should learn to address a broader, general audience.
This means that you will not take certain things for granted as you write and ensure
that you provide good contextual information designed to help your readers gain clear understanding. 

For example, while your professor knows the authors and readings he or she assigned in the class
very well, when you reference them in your writing you should take care to
introduce them, thus addressing a more expansive audience who may not
be familiar with the texts or authors in question. To illustrate, consider the following two
sentences:

\newpage

\textbf{1. As we talked about in class, Freire argues that banking can be undemocratic and oppressive.}

\begin{itemize} \item \hloy{This sentence assumes that the audience knows
certain things, namely Paulo Freire, his essay, and class discussions}. However,
writing the sentence in this way excludes everyone who is not taking part in
this particular class. Imagine the confusion you would experience after reading
this sentence if you had not taken part in the discussion of this piece of
writing. You might wonder: What is "banking"? You mean my credit card company is oppressing me? Who is this Freire guy? Is he some authority I should trust? Where did he make this argument?
\end{itemize}

\textbf{2. In an essay entitled "The `Banking Concept' of Education," author and educator Paulo Freire argues that a widely practiced form of schooling that he terms "banking education" is oppressive and undemocratic.}

\begin{itemize}

\item \hloy{The second sentence, however, attempts to include a much larger
audience by carefully introducing important contextual information}. By
providing the author's full name, his profession, the essay title, and the
definition of key terms, the audience will feel that they are being addressed by
your writing, rather than excluded. Further, they will gain important contextual
information that is needed for understanding.

\end{itemize}
